% This is one line latex comment

\section{Result \& Analysis}

To derive the relationships among the flow elements, I have conducted a series of experiments.
In this section, I presented the obtained results and derivation of the mathematical model.

\subsection{Effect of Potential Difference}

\begin{figure}[!t]
  \centering
  \begin{subfloat}[][Potential difference-current data and linear regression model: $y = 1.3677x + 0.3404, R^2 = 0.731$. $y$ denotes current, $x$ denotes relative potential difference.]
    {
      \includegraphics[width=2.5in]{./data/voltage-plot}
    }
  \end{subfloat}

  \hfill

  \begin{subfloat}[][Studentized residuals of linear regression model.]
    {
      \includegraphics[width=2.5in]{./data/student-voltage-plot}
    }
  \end{subfloat}

  \caption{Relationship between potential difference and intensity of current.}
  \label{plot:voltage-current}
\end{figure}

Figure \ref{plot:voltage-current}(a) shows the observed relationship between potential difference and intensity of current.
From the result, I constructed a simple, linear regression model.
As $R^2$ value of the linear model indicates, the constructed model well describes about 73\% of the data.
Also, to check whether this model fits into linear model, I performed a studentized residual analysis.
The analysis is shown in Figure \ref{plot:voltage-current}(b).
The proposed model well fits with the linear model because the studentized residuals show a random distribution centered at zero and most of the points are in range from -2 to 2.

As the linear model implies, the current flowing in a Galvanic circuit is directly proportional to the potential difference applied to the circuit.
So, this finding would be used in Section \ref{subsection:derivation}.

\subsection{Effect of Wire Length}

Relationship between current intensity and wire length is depicted in Figure \ref{plot:length-current}.

\begin{figure}[!t]
  \centering
  \begin{subfloat}[][Wire length-current data]
    {
      \includegraphics[width=2.5in]{./data/length-plot}
    }
  \end{subfloat}

  \begin{subfloat}[][Inverse of wire length-current data and linear regression model: $\frac{1}{y} = 0.0203x + 0.0416, R^2 = 0.725$. $y$ denotes current, $x$ denotes wire length in meters.]
    {
      \includegraphics[width=2.5in]{./data/inverse-length-plot}
    }
  \end{subfloat}

  \hfill

  \begin{subfloat}[][Studentized residuals of linear regression model.]
    {
      \includegraphics[width=2.5in]{./data/student-inverse-length-plot}
    }
  \end{subfloat}

  \caption{Relationship between wire length and intensity of current.}
  \label{plot:length-current}
\end{figure}

\subsection{Effect of Cross-sectional Area of Wire}

\subsection{Mathematical Modeling of Electric Flow}
\label{subsection:derivation}
