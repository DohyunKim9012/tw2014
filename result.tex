% This is one line latex comment

\section{Result \& Analysis}

To derive the relationships among the flow elements, I have conducted a series of experiments.
In this section, I presented the obtained results and derivation of the mathematical model.

\subsection{Effect of Potential Difference}
\label{subsection:voltage}

\begin{figure}[!t]
  \centering
  \begin{subfloat}[][Potential difference-current data and linear regression model: $y = 1.3677x + 0.3404, R^2 = 0.731$. $y$ denotes current, $x$ denotes relative potential difference.]
    {
      \includegraphics[width=2.5in]{./data/voltage-plot}
    }
  \end{subfloat}

  \hfill

  \begin{subfloat}[][Studentized residuals of linear regression model.]
    {
      \includegraphics[width=2.5in]{./data/student-voltage-plot}
    }
  \end{subfloat}

  \caption{Relationship between potential difference and intensity of current.}
  \label{plot:voltage-current}
\end{figure}

Figure \ref{plot:voltage-current}(a) shows the observed relationship between potential difference and intensity of current.
As $R^2$ value of the linear model indicates, the constructed model well describes about 73\% of the data.
Then, to prove that this model well fit to the data, studentized residual analysis is performed.
The result of this analysis is shown in Figure \ref{plot:voltage-current}(b).
The proposed model well fits with the linear model because the studentized residuals show a random distribution centered at zero and most of the points are in range from -2 to 2.

As the linear model implies, the current flowing in a Galvanic circuit is directly proportional to the potential difference applied to the circuit.
So, this finding would be used in Section \ref{subsection:derivation}.

\subsection{Effect of Wire Length}

\begin{figure}[!t]
  \centering
  \begin{subfloat}[][Wire length-current data]
    {
      \includegraphics[width=2.5in]{./data/length-plot}
    }
  \end{subfloat}

  \begin{subfloat}[][Inverse of wire length-current data and linear regression model: $\frac{1}{y} = 0.0203x + 0.0416, R^2 = 0.725$. $y$ denotes current, $x$ denotes wire length in meters.]
    {
      \includegraphics[width=2.5in]{./data/inverse-length-plot}
    }
  \end{subfloat}

  \hfill

  \begin{subfloat}[][Studentized residuals of linear regression model.]
    {
      \includegraphics[width=2.5in]{./data/student-inverse-length-plot}
    }
  \end{subfloat}

  \caption{Relationship between wire length and intensity of current.}
  \label{plot:length-current}
\end{figure}

Relationship between current intensity and wire length is depicted in Figure \ref{plot:length-current}(a).
From Figure \ref{plot:length-current}(a), I made a conjecture that the current intensity and wire length is inversely proportional.
To prove this to be true, further analysis is performed with the inverse of current intensity and the wire length.
The corresponding plot is shown in Figure \ref{plot:length-current}(b).

With inverse value of current intensity and wire length, the same approach from Section \ref{subsection:voltage} was used.
The linear model is shown in \ref{plot:length-current}(b).
It explains about 72.5\% of the obtained data and showing an apperant linear relationship.
Studentized residuals are distributed randomly around zero and most of them are in range between -2 and 2.

Figure \ref{plot:length-current} also includes the result from using zero-length wire.
In this case, this zero-length is made with the direct connection of the two terminal lines of thermocouple.
An electric flow is also found in such condition and this fact implies that some variable other than length of wire exists in this experiment.
This variable is not changed throughout the experiment, because the only changed variable is length of wire.
In that point, this variable can be regarded as a constant, and it can be implied that this constant is from experiment setting.
I think this constant is related to internal condition within the experiment equipments, especially the thermocouple.
This is because the thermocouple itself also contains its own wires and these wires are also included in the total path around circuit.
Also, this constant is included in the proposed linear model.

According to the data shown in \ref{plot:length-current}, the intensity of current flowing through the Galvanic circuit is inversely proportional to the length of wire within the circuit plus some constant caused by experiment setting.
This finding was applied to further discussion in Section \ref{subsection:derivation} for derivation of mathematical modeling.

\subsection{Effect of Cross-sectional Area of Wire}
\label{subsection:cross-section-current}

\begin{figure}[!t]
  \centering

  \includegraphics[width=2.5in]{./data/cross-section-plot}

  \caption{Relationship between cross-sectional area of wire and current. Constructed linear regression model: $y = 3.5415x + 3.3932, R^2 = 0.370$. $y$ denotes current, $x$ denotes cross-sectional area of wire ($mm^2$).}
  \label{plot:cross-section-current}
\end{figure}

\label{subsection:derivation}
